%%%%%%%%%%%%%%%%%%%%%%%%%%%%%%%%%%%%%%%%%%%%%%%%%
%%%%%%%%%%%% cap: Conclusions %%%%%%%%%%%%%%%%%
%%%%%%%%%%%%%%%%%%%%%%%%%%%%%%%%%%%%%%%%%%%%%%%%%

\chapter{Post-analysis remarks}\label{cap:remarks}


\section{Hypothesis vs Results}

Before testing the tools in cause, the expectations were that Static Code Checkers would be trained in a way that would always suggest similar, if not the same changes for code restructures and possible flaws. In regards to vulnerabilities, given how the test samples were short code snippets, a fairly high detection rate was to be desired with a fairly correct vulnerability type detection, however, as it turns out, the hypothesis was slightly off. Lastly, in regards to threat detection, the expectations of them being able to detect threats were minimal.\\

\noindent Upon testing, it was quite clear that Static Code Checking tools were able to improve code readability, performance and fix potential problems while being able to maintain the pre-imposed coding ethics. Continuously, these tools did not arise to expectations for vulnerability detection and vulnerability type detection, however, they did not fail to detect a single common vulnerability and neither any form of deadly vulnerability. Finally, the tools tested were unable to detect any kinds of threats and applications with a malicious intent, just as expected.


\section{Is SCC worth it?}

While tools may not always be perfect at their job, neither will humans, especially when such a broad topic is the main focus of it. So, granted the prior results, namely in:  \ref{scaResults} \ref{VSCAResults} and \ref{TDTSCResults}, we can see that the tools outperformed the humans by quite a big margin on average, however, that statement may change based from person to person. Henceforth, Static Code Checking is indeed worth it as a mere copilot in development.


\section{Final remarks}

Given the undeniable results listed in previous parts of the paper, there is great evidence in supporting the fact that Static Code Checking, alternatively named Static Code Analysis, is worth the investment in software development. Treated as copilots, Static Code Checking tools have proven to be of great use in early issue and vulnerability detection together with improving code quality. Through these means, not only will the work of programmers become significantly easier, but also the productivity shall increase. The consistency in coding standards and maintainability will be of great aid to anyone attempting to better understand previously written code.\\  

\noindent Subsequently, various tools have proven themselves as possibly valuable assets in a programmer's kit due to their capabilities in spotting possible errors and vulnerabilities in code of all types, removing a handful of security risks in larger scale projects, and saving a lot of time for the developers in fixing the issues by giving good hindsight over how to approach the problem and how to solve it.\\

\noindent Ultimately, Static Code Checkers have not had a great performance in spotting and detecting any form of malicious apps and henceforth it is not advised to rely on them in order to keep your device safe and secure. When being given open-source code, the safer option would be building the application and resorting to an antivirus to detect whether or not the application will be a threat to your device or not.

\section{Future Research}

With the daily increase in the capabilities of Artificial Intelligence and overall possibilities for programmers, as a future continuation of this paper, various extensions will be compared amongst each other, and together with that, a more in-depth look through various AIs such as ChatGPT will be done, in order to test whether or not they will surpass the capabilities of Static Code Checking tools. \\

\noindent Furthermore, tools will be compared against each other in fields of mutual expertise such that we will be able to determine which tool would outperform the others and ultimately which would be more advisable for use in cases of large-scale teams, small-scale teams and individual projects. 