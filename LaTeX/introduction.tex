%%%%%%%%%%%%%%%%%%%%%%%%%%%%%%%%%%%%%%%%%%%%%%%%%
%%%%%%%%%%%% cap: intro %%%%%%%%%%%%%%%%%
%%%%%%%%%%%%%%%%%%%%%%%%%%%%%%%%%%%%%%%%%%%%%%%%%

\chapter{Introduction}\label{cap:intro}


\section{Motivation}\label{sect:motivation}

As technology is becoming more prominent in everyone's daily life, malicious intending people also gradually start targeting their victims through virtual means. In the field of computers, it is easier to get lost or lose track of one's senses and fall victim to these malevolent actions. This thesis takes on the incentive of verifying the competence of key-tools within a programmer's tool set and their accuracy in detecting threats. The tools in cause are categorised as Static Code Checkers and their purpose is to detect vulnerabilities, bad structure of code or any other mistakes that a programmer can make; these tools are made in order to help the users develop cleaner code, simplify their code's complexity, identifying potential vulnerabilities and lastly to attempt improving resource utilisation. There are various existing tools capable of performing such actions, however, there are not many in depth comparisons over these tools, and especially on different type of data; henceforth, my motivation is trying to determine the difference between tools in their own areas of expertise, and subsequently, in detecting threats. By doing so, conclusions will be able to be drawn in regards to which tools are most recommended or advised to use for a more multi-purpose aim.\\

\noindent To add to the above, comparisons between different tools will also be run on 2 other types of tests, in order to detect the usefulness of them and perhaps be able to tell if there are more advanced static code analysis tools available out there that can be integrated into larger scale projects. The reason for the addition of the latter is solely due to the fact that many companies have products whose source code is not particularly easy to follow, does not fit the code standards, or has possible flaws. 